\documentclass{article}

\usepackage{fullpage}
\usepackage{amsmath}
\usepackage{amssymb}
\usepackage{amsthm}
\usepackage{graphicx}

\begin{document}

\begin{figure}[h]
  \centering
  \includegraphics{comparing-updates-domain.eps}
  \caption{A square domain $\Omega$ for comparing triangular
    updates.}\label{fig:comparing-updates-domain}
\end{figure}

Let $u : \Omega \to \mathbb{R}_{\geq}$ denote the solution to the
equation $||\nabla u|| = 1$, and assume that $u$ satisfies the
boundary condition:
\begin{align}
  u(x) = 0, \qquad x \in [x_0, x_1] \cup [x_1, x_2]
\end{align}
Further, assume that the sides of the square domain $\Omega$ are of
length $h > 0$. Let $[x_0, x_2]$ be parametrized by
$\lambda \in [0, 1]$ so that $x \in [x_0, x_1]$ can be written
$x_\lambda = (1 - \lambda) x_0 + \lambda x_1$. Then, by an application of Huygen's
principle:
\begin{align}
  u(x_\lambda) = h \min(\lambda, 1 - \lambda), \qquad \lambda \in [0, 1], \qquad x_\lambda \in [x_0, x_2]
\end{align}
Since $u$ is symmetric about the diagonal
$[\hat{x}, x_1] \subseteq \Omega$, both diagonal updates yield the
same estimate for $\hat{u}$---call these values
$\hat{u}^{(\operatorname{diag})}$. Then:
\begin{align}
  \hat{u}^{(\operatorname{diag})} = u_{\lambda^*} + \hat{s} ||\hat{x} - x_{\lambda^*}|| = h,
\end{align}
since $s \equiv 1$, $\lambda^* = 0$, and
$||\hat{x} - x_{\lambda^*}|| = h$. Along the same lines, we have:
\begin{align}
  \hat{u}^{(\operatorname{adj})} = \frac{\sqrt{2}}{2} h.
\end{align}
Hence:
\begin{align}
  \hat{u} = \min(\hat{u}^{(\operatorname{diag})}, \hat{u}^{(\operatorname{adj})}) = \frac{\sqrt{2}}{2} h.
\end{align}
By another application of Huygen's principle, it is straightforward to
see that $u(\hat{x}) = h$.

\end{document}

%%% Local Variables:
%%% mode: latex
%%% TeX-master: t
%%% End:
